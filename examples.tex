\subsection{Social Capital}
This paper will consider the following two definitions of social capital, as defined by Robert D. Putnam: "The information, trust, and norms of reciprocity inherent in one's social networks" \cite{Putnam1995} And Nan Lin's: "Individual social capital is conceptualised as accessible resources embedded in the social structure (..)" \cite{Nan2001}. The two compliment each other in  that Nan Lin describes the value of ones human social connections with the overall term 'resources', while Putnam's 'information, trust and norms of reciprocity' should be seen as a subset of the 'resources' which Nan is referring to. 

\subsection{Individuals with low social capital} In order to describe an individuals level of social capital, three quantities were defined by Zhao Y \cite{Zhao2002}. Which will be the baseline measures of social capital. The tree measures are:
\begin{itemize}
    \item Network Size
    \item Network Density
    \item Embedded network resources
\end{itemize}Network size refers to the number of social relations an individual has. Network density accounts for the closeness of the social relations. The embedded network resources measure, describes the amount of resources in the network, where resources are defined as in the previous section by Nan Lin and Putnam. Therefore a person of low social capital, will be a person with a relatively low values in the tree measures stated above.

\subsection{Language skills for job market}
The term linguistic performance, as coined by Chomsky will be used as the definition of language skills; "the actual use of language in concrete situations" \cite{Chomsky1965}. This definition presents a more holistic view of language and takes into account differences in dialect, accent, cultural background and knowledge of slang, when assessing an individuals competence in said language. When assessing linguistic performance, the term "cultural disconnect" can be used to describe difficulties in understanding a language spoken in a strong dialect, or the bilateral meaning of i.e 'job' as an employment opportunity and a physical act \cite{Thakkar}.

\subsection{Bias}
The term bias refers specifically to the notion of cognitive bias, introduced by Tversky and Kahneman \cite{Kahneman2002}, and more specifically biases that affect judgement and decision-making. Differences in human judgement and decision-making are explained by involving mental shortcuts to provide estimates of uncertain outcomes, such as the possibility of an applicant becoming a valued employee of the company. These biases exist on both an implicit level (where the bias are upheld by the individuals) and on a systemic level (where the bias are maintained by institutions), and influences all aspects of life, including the job-seeking process \cite{MediumArticle}.

\subsection{Mobile interaction tool}
A piece of application software designed to run on a computer or mobile device, allowing for interaction between different devices, thereby mediating communication through different channels from different stakeholders point of view \cite{Application}. This type of software is included under the much larger umbrella term ICT (Information and Communications Technology) \cite{ICT}.

\subsection{Accessibility to the job market}
Ease of accessing the job market based on factors such as education level, geographical location or discrimination based on ethnicity, gender, cultural background or religion \cite{Paulsen2017}.

\subsection{Digital literacy}
The ability, for an individual, to understand and use digital information as defined by Paul Gilster \cite{gilster1997digital}. When discussing digital literacy, the subterm digital exclusion describes the lack of technology resources and access for individuals in poverty, minorities or language learners \cite{Exclusion}. 

Theory:

\subsection{Intro}
Studies have been carried out in Denmark identifying a bias in the job-seeking process, with immigrants with a middle eastern sounding name having to write 52\% more applications to secure a job interview \cite{Politiken}, causing individuals to change their name to sound more European in order to secure a job interview \cite{Afrodanskere}. Minority women who wear a scarf have to write 60\% more applications to secure a job interview, when compared to ethnically Danish women, and 43\% of young individuals with an ethnic minority background have faced discrimination within the last year \cite{Samvirke}. Based on the premise that bias exist on numerous levels (i.e definition of concepts) we highlight the factors which contribute to bias' toward refugees and immigrants, being;
\begin{itemize}
  \item Relocation and its consequences
  \item Social capital
  \item Digital illiteracy
  \item Cultural barriers
  \item Language skills
\end{itemize}
 The factors contributing to bias, can be at a systemic level, implicit level, or both. Lastly in the literature review, a section will be covering literature which deals with attempts on decreasing or removing bias in the job-seeking process, by means of digital technologies.

\subsection{Relocation and its consequences}
As refugees and immigrants relocate to other nations, they are leaving behind their social capital\cite{Almohamamed2018}(n=14). Along with lowered language skills\cite{Rockwool2013}, compared to the host population, they are at a significant disadvantage when entering the job market, where both elements are vital, in terms of them acquiring employment\cite{putnam2000bowling}\cite{aldridge2002social} and succeeding in the job. The total sum of these groups problems creates a significant deficit, in terms of integrating both into the job market and society. The individual contributions of the problematic factors will be investigated further in the subsequent sections.

\subsection{Social capital}
Refugees leave behind family, friends, and coworkers when they choose to immigrate to another country. This effectively decreases their Network size and density, and therefore also the number of embedded resources. 
Shuttleworth et al.\cite{shuttleworth2008incapacity} and Green et al.\cite{green2011job} describes the ability to use social capital within ones network, in the pursuit of information about, and introductions to, potential employers as one of the most effective strategies when seeking employment. Specifically for Denmark, an immigrants individual disposition and social context has been identified as determining factors when analyzing the labour market integration experiences of said individual\cite{Sirius}(Sirius report: n=27,M=11,F=16). In the report by Bontenbal\cite{Bontenbal2019}, conveying the views of danish ministries, municipalities, employers' confederation and more, it is concluded that \textbf{\emph{successful labour market integration is highly dependant on a social network.}} The report emphasises the importance of mentors from the host community, and highlights mentors from the host community or workplace as the consensus best drivers of labour market integration. The report concludes that immigrants networks have an ambiguous impact on labour market integration. Mentors from the immigrant network is highlighted as potential helpers in overcoming language barriers, but might also enforce job-seeking within sectors, which are not corresponding to the individuals educational background. The report also highlights the negative effects of so-called 'ghetto'-formation as a parallel society where little to no interaction between different cultures is present. 


\subsection{Digital illiteracy}

A vast number of initiatives exist in European nations aimed at increasing the digital literacy of migrants \cite{EU2019}, due to this group being more affected by the phenomena of digital exclusion than the rest of the European population \cite{EU2006}, further exacerbating the existing inequality. The scope of the issue remains unclear however, as no large-scale studies determining the actual digital literacy of migrants in comparison to native-born citizens has been publicized to our knowledge.

\subsection{Cultural barriers}

Cultural barriers between traditional western cultures (i.e USA \& Australia) and cultures from which refugees migrate (traditionally middle-eastern, African and Asian cultures) is a widely covered topic in the literature. Studies from Australia\cite{Almohamed}(n=14) and USA\cite{Brown2016}(n=11 qualitative study, n=19 field study) suggest that cultural barriers between the cultural groups in question exist, and that they influence the overall integration process. Literature further covers the cultural barriers between the cultures of refugees, and the technologies created by western culture\cite{Almohamamed2018}. In terms of retaining or creating social capital among people of the same cultural group, barriers also exist. Studies carried out in the western countries, and in Denmark specifically suggests that the concepts of 'pride' and 'dignity' withholds refugees from accessing certain services or aids, and also making them withhold intimate information about i.e their poor employment situation in the host country from their network\cite{Sirius}. This is due to the lack of employment and the use of social aids/services, in some cultures are considered shame full\cite{Brown2016}. In terms of social capital, this hinders the mobilization of embedded network resources, since problems of individuals residing within the network are not communicated, and therefore cannot utilize the network. Bontenbal\cite{Bontenbal2019} mentions three key culturally rooted problems when migrants are integrated in the danish labour market. These three problems are punctuality, participation in social life at the work place, and mixed gender workplaces. The problem of punctuality takes different forms, depending on which migrant group we a re looking at. Some lack knowledge of the importance of being on time in Denmark, some lack knowledge of what do to if you are late. One case did not show up to work, due to the fact that she was late, and expected to be replaced by another employee, since that was standard procedure in her country of origin. The participation in social life is lacking due to either lack of language skills, knowledge regarding social norms, or both. A subject expresses the participation in social activities like eating lunch together or social drinking as a problem. The mixed gender workplaces are for some migrants an unknown phenomena, and can therefore be hard to integrate within. \textbf{\emph{These culturally dependant factors were found to hinder individuals attempting to integrate into the danish labour market.}}


\subsection{Language Skills}

In terms of connecting language skills (or the lack thereof) to the employment situation of many refugees and immigrants, the literature suggests a negative influence in many ways. As for the linguistic competence of refugees, illiteracy (both in their own language and in the language of their host nation) traditionally exists in higher rates than for their respective host nations\cite{Marshall}\cite{UN}. This illiteracy connects with literature suggesting that access to job-services through ICT \cite{AhmedKhan2017} is a challenge for this group. Almohamed et al.\cite{Almohamamed2018} highlighted that the language barrier hindered refugees from rebuilding social capital, through connecting with the host communities. As proposed by literature previously\cite{shuttleworth2008incapacity}\cite{green2011job} such links are ranked as the second most effective mean of getting employment. In that way the language barrier infers indirectly on the employment opportunities of refugees\cite{Hauck2014}. In terms of the sub-term of language skills in 'linguistic performance' the study by Thakkar \cite{Thakkar} highlights the difficulty 'cultural disconnects' can infer on the access to online learning services in a second language. This can i.e be the way certain English words can have two meanings, dependent on the context and culture using the word. The two meanings of 'job' interpreted by traditional western cultures as meaning predominantly one thing, and in the example Indian culture, meaning something else. The danish language schools focus on "work danish"\cite{Sirius} thereby causing individuals to experience this cultural disconnect, when attending educational institutions or engaging with ethnic Danes in a social context. Others feel pressured by the language requirements in order to access educational services, leading them to forfeit this ambition\cite{Sirius}. Social partners concerned with refugees' and immigrants labour market integration in Denmark, highlight language skills, and the recognition of language skills, as one of the main barriers to labour market integration\cite{Bontenbal2019}.In the same report by Bontenbal, the danish government, ministries and official national institutions working with refugees, does not recognize skills (i.e professional and language) as a barrier to the labour market. Bredgaard et al.'s meta study concludes that \textbf{\emph{language skills is a barrier to labour market integration in Denmark.}}\cite{Bredgaard2018}  

\subsection{Existing solutions}

With the literature indicating a general problem in refugees and immigrants achieving successful employment, due to reasons stated above, the necessity for a solution which can bridge the gap between the job market and the group is needed. ICT services has been deployed in different instances to counter these problematic tendencies. The study by Alam, K. et al. concludes that ICT's and digital technology were seen as a vital tool in acquiring job opportunities and reestablishing social capital in a general sense\cite{alam2015digital}(n=28,f=18,m=10). The study by Brown, G. et al. \cite{Brown2016} investigated the potential for a 'human-in-the-loop' translation platform trying to connect stakeholders without a common language. They concluded that the tool, enabled further communication between mentors and refugees, and by the translation being facilitated by former refugees themselves, cultural disconnects could also be solved. Additional studies have implemented similar digital technologies for refugees in order for refugees to bridge language gaps and connect with the host society, in which they are deployed \cite{Hirsch2004}(n=226, guides=26,users=200)\cite{Baranoff}(n=9,M=1,F=8). One thing to notice, is that the technologies found in the literature, mainly deal with bias on the systemic level, and not many solutions for reducing implicit bias in ICT, were found. The literature calmly suggests a trend of successful adoption and effect of technologies, in assisting refugees problems of employment. Whether it translates to danish circumstances will remain to be proven, since literature concerning stakeholders in Scandinavia or Denmark is limited, as per our knowledge.

\newpage

Methods: The methods used throughout the design process can be found in the table below, structured around the double diamond framework. The Seed uses co-design\cite{Codesign} and inclusive design\cite{InclusiveDesign} as the design framework in the development of their platform, and these methods will therefore influence the process presented in this thesis.

Furthermore, initial data was provided by the seed, who had done previous interviews with relevant stakeholders. This data was given to us prior to the projects official start. The methods of collection and quality of the data deemed it worthy of being taken into the project, since it was from relevant stakeholders.

\newcolumntype{b}{X}
\newcolumntype{s}{>{\hsize=.5\hsize}X}

\begin{tabularx}{\textwidth}{|>{\columncolor[gray]{0.9}} s | s | b | b | b |}


    \multicolumn{2}{c}{\textbf{ }} \\
    \hline
    \hline
    \rowcolor[gray]{0.9}
     & Methods    & Participants and any related methods    &  Activities & Expected outcomes        \\
    \hline
 \hline
             Project Management & Scrum \cite{Scrum}                              & The Seed                                                                                             & Used as the time and responsibility management tool       &                                                                                           \\ \hline
             \hline
Design Framework   & Double Diamond \cite{DoubleDiamond}                    &                                                                                                                         &   &                                                                                                                                                                                                                                            \\ \hline
\hline
Discover           & Proto-Personas      \cite{Protopersonas}                           & The Seed                                                                                                             &                     Proto Personas for a job seeker and employer will be developed.   & A series of personas that represent our different target users                                                                            \\ \hline
                  
                   & PACT       \cite{PACT}                        & Brainstorming will be used as a part of this method                                                                   & Brainstorming used to scope out the possible variety of users.  &  Creating an overview of different users and their needs                                                                                         \\ \hline
                    & Actor Worlds \cite{Arena} &                                                                                                                         & Define actorworlds and development arena of product.
                    & Identifying connecting actors and actor worlds
                    \\ \hline
                    & Contextual Interviews   \cite{ContextualDesign}           & Immigrant job seekers (4) and employers (2)                                                                                             & Contextual Interview with targeted users will be carried out. & Relevant insights from first hand sources
                    
                    \\ \hline
                    & Usability Testing  \cite{ContextualDesign}           & Immigrant job seekers (2), Danish students (4) and employers (2)          & Tests will be set up to determine the usability of the prototype & Measurable performance data of prototype
                    
                    \\ \hline
                    \hline
                   Define & Personas  \cite{Personas}                         & The Seed (Learning Content Designer), ITU Students, Proto-Personas will be used as part of this method                                                          & Proto Personas are specified based on collected data. & A more accurate and encompassing set of personas                     \\ \hline
                   & KJ Technique    \cite{ContextualDesign}                     &                                                                                                                         &  Team members will individually write down ideas, and the individual ideas are then presented, clustered and voted on.    & Selection of relevant ideas and points of interest for development.                    \\ \hline
                     & Wallwalk   \cite{ContextualDesign}                        &                                                                                                                         & Creating an overview of collected empirical data using a wall to visualize it. & Structuring of collected empirical data, and identification of relevant insights.                                                                    \\ \hline
                   & Function-Means Tree   \cite{EngineeringDesign}             &                                                                                                                         & The overall functions of the service are broken down into its underlying basic functions illustrated as a tree.   & Identification of fundamental functions necessary for prototype.                                  \\ \hline
                   & Morpho-logy  \cite{EngineeringDesign}                       &                                                                                                                         & Related functions are morphed together to reduce complexity of service.   & A series of concepts that can be measured against one another.                                                                          \\ \hline
                    & WOM (Weighted Objectives Matrix)  \cite{EngineeringDesign}  &        The User-Experience Honeycomb will be used to define the parameters each concept will be weighed on \cite{UXHoneycomb}.                                                                                                                 & Different aspects of the service are weighted according to their importance. The concepts created by the Morphology are then held up against one another, based on these weighted aspects. & Selection of a single concept that will be further developed.                                                     \\ \hline
                    \hline

                Develop   & Story-boarding   \cite{Storyboard}                   &                                                                                                                         & Storyboards for interaction with app are created.    & An overview of the interaction elements requirements is created.                                                                                               \\ \hline
                   & Paper Prototypes \cite{Paperprototype}                  &                                                                                                                         & Paper prototypes of digital tabs are created    & Paper prototypes that can be used for role-playing, Co-creating Workshop and Usability testing.                                      \\ \hline
                    & Role-Playing  \cite{Role-Playing}                     &                                   & Role-play is carried out throughout the creation of the  prototype to ensure the UI is intuitive. & A more intuitive prototype (and thereby improved final concept)                                     \\ \hline
                   & Co-Creation Workshop  \cite{Codesign}             & Representative group of immigrant job seekers (4, based on our developed personas), employers (2), Paper Prototypes/Click-Through Prototypes will be used as part of this method & Co-Creation workshops are set up with a varied group of targeted users, who are then presented with Paper Prototypes (Or Click-Through Prototypes).  & Performance analysis of prototype. Deeper insights into the requirements for the product, and areas of focus for next iteration of the prototype.
                   \\ \hline
                   \hline
Deliver            & Click-Through Prototype  \cite{Clickthrough}          &                                                                                                                         & A clickable prototype of the application is developed      & A testable clickable prototype, that will serve as a final concept.                                                                                         \\ \hline
 
\end{tabularx}

\newpage

\subsection{Overall project structure}
The project is structured around the design framework named 'Double Diamond'\cite{DoubleDiamond}.

\begin{figure}[H]
\caption{Double Diamond model}
\centering
\label{DoubleD}
\includegraphics[width=\textwidth]{Images/Methods/Double-diamond.jpg}
\end{figure}

The framework consist of four different phases namely: 'Discover','Define','Develop', and 'Deliver'[\ref{DoubleD}], each having either a converging focus or diverging focus. These four steps will be completed in an iterative manner described in the figure below[\ref{Iterations}]:

\begin{figure}[H]
\caption{Iterations of Double Diamond over time}
\centering
\label{Iterations}
\includegraphics[width=\textwidth]{Images/Methods/Iterations.PNG}
\end{figure}

Each iteration will end with a prototype, which will be the case of the next iteration round of design, ending up with a final click-through prototype November 27th.

\subsection{Application of methods}
This following section details the specific application of the before mentioned methods, and will outline the purpose of the methods-selection and the execution of the method. The section will be structured so that, iterations 1 through 3, each containing a separate round of the overall double diamond steps [\ref{DoubleD}] where the Discover and Define phases will be covered together, and Develop and Deliver together. The focus of this project was on creating a UI for the section of the platform from when an applicant have applied for a position and can face bias (marked in grey), as shown in the figure below. 

\begin{figure}[H]
\centering
\caption{Simplified overview of The Seed's platform}
\label{click-through1}
\adjustbox{trim={.0\width} {.15\height} {0.0\width} {.3\height},clip}
{\includegraphics[width=\textwidth]{Images/Methods/TheSeedOverview.jpg}}
\end{figure}

\subsection{Discover \& Define}
The Discover \& Define phases includes the following methods used throughout the iterations when needed[\ref{DiscoverAndDefine}]: 

\begin{figure}[H]
\caption{Methods used in the Discover and Define phase}
\centering
\label{DiscoverAndDefine}
\includegraphics[width=250pt]{Images/Methods/discover and define.PNG}
\end{figure}

\subsubsection{1. Iteration}
Throughout the first iteration of Discover and Define, the following methods were applied:
\begin{itemize}
    \item Proto-Personas
    \item PACT
    \item Actor worlds
    \item Contextual interview
    \item Function Means Tree
    \item Morphology
    \item KJ Technique 
    \item Wall Walk
    \item User Specification
    \item WOM (Weighted Objectives Matrix)
\end{itemize}

Each methods application and goal, will be detailed in the section below:

\begin{itemize}
    \item \bf{Proto-Personas:}
\end{itemize}

The scope given from the The Seed, dictates two types of stakeholders, the 'employer' and the 'job-seeker'. For the first iteration of design, we used the proto-personas-method to suggest a diverse field of personas based on the literature both covering employers and job-seekers. The purpose of this process was to both give an indication to the company, which handled stakeholder recruitment, who to go for, and also to get a visual grasp of the diversity within the populations in question. The method suggested a framework in which you fill out 4 boxes containing:
\begin{itemize}
    \item A fictional name
    \item Biographical Information
    \item Character Traits
    \item Values
\end{itemize}

These four dimensions should encapsulate and describe archetypes of the different types of individuals within the titles 'employer' and 'job-seeker'.

\begin{itemize}
    \item \bf{PACT:}
\end{itemize}

The 'PACT'-method takes the stakeholders described in the proto-personas outcome, and describes them in terms of 4 dimensions:
\begin{itemize}
    \item People
    \item Activities
    \item Context
    \item Technologies
\end{itemize}
This method was applied to visualize the knowledge both gained from The Seed, and the information gathered through the literature review about these individuals we are to design for. The visualization will allow us as designers to further step into the world of the stakeholders, and internalize their values with the goal of successfully design features so that it meets the stakeholders needs.

\begin{itemize}
    \item \bf{Actor Worlds:}
\end{itemize}

The 'Actor Worlds'-method can be used to describe the landscape of actors surrounding a product, service or company, and to predict implications the introduction of a product through the 'development-arena' will cause to these 'actor worlds'. For the scope of this project, we applied the method in order to describe the stakeholders surrounding The Seed, and its product, and not try to predict the implications through the development arena.

\begin{itemize}
    \item \bf{Contextual Interviews:}
\end{itemize}

The method was only applied on the stakeholder 'employer' in the first iteration, due to problems with recruiting subject from the 'job-seeker' bracket before the deadline of the first prototype. The employer had prior to us executing this interview, given input to The Seed, which was valuable to us as designers, and the interview we conducted sought to elaborate on this input. The interview was semi-structured, with questions seeking to discover the values of the employer, and the intangibles of both a good job-seeking process and a bad one.

\begin{itemize}
    \item \bf{KJ-Technique:}
\end{itemize}

The 'KJ-Technique'-method sought to structure the chaotic post-its produced from the contextual interview, along with the post-its created from a brainstorm, attempting to synthesize similar post-its from the 'job-seeker' point of view. This was achieved by us, the designers, silently coupling post-its which we felt were linked to each other. One would start by taking a poster, putting it on the board, and then we would go through the other post-its silently, until no other post-its could be put on the board of relevance. Thereafter, a structure would be assigned to each cluster, depending on how the posters fit together thematically.

\begin{itemize}
    \item \bf{Wall Walk:}
\end{itemize}

The 'Wall Walk'-method sought to generate so-called 'hot-ideas' which are concrete design suggestions based on the categorized input from the 'KJ-technique'-method. We addressed each category by doing a two-minute silent brainstorm for both of us, creating two types of post-its:
\begin{itemize}
    \item Solutions (Green) [\ref{Wall}]
    \item Problems (Blue) [\ref{Wall}]
\end{itemize}
The silent brainstorm was conducted within the structure described in the corresponding section in the analysis and results-section.

\begin{itemize}
    \item \bf{User Specification:}
\end{itemize}

The User Specification is a method for detailing the requirements and criteria which a solution needs to live up to, in order to be considered 'good'. The requirements and criteria were based on the basic principles from the UX-honeycomb\cite{UXHoneycomb}, and basic UI/UX design principles. The method will infer on the WOM-method, where the different factors in the User specification, will be the evaluation criteria of the WOM-method, which seeks to evaluate the 'goodness' of a concept, based on the criteria of the user specification.

\begin{itemize}
    \item \bf{Function Means Tree:}
\end{itemize}

The 'Function Means Tree'-method, was applied to define, what basic function the systems should provide, through asking structured questions detailing further and further the functionality of the system. A certain level of the Function Means Tree (highlighted in green letters in figure [\ref{Tree}]) will dictate the columns of the morphology matrix, which is also known as the structure of the morphology matrix.

\begin{itemize}
    \item \bf{Morphology:}
\end{itemize}

The 'Morphology'-method was applied to create a framework for concept-creation based on the input solutions from the Wall Walk and the Function Means Tree. Through the columns the different solutions were subdivided, as they fit, and further brainstorms were conducted, for each column, in order to see whether more appropriate solutions could be added to the existing ones. After the brainstorm (one for each separate column), 5 different titles for concepts were proposed to capture different extremes of the different functions. The concepts were thereafter drawn from the existing functions with the goal of trying to create coherence between the functions and the objective of the concept.

\begin{itemize}
    \item \bf{WOM (Weighted Objectives Method):}
\end{itemize}

The 'WOM'-method was applied in order to evaluate the concepts from the morphology, based on quantitative measures of the concepts 'goodness' with a structure from the User Specification. To determine the weights we are using a so-called 'Pugh'-approach, where the designers discuss the factors one by one and rank them in a matrix, giving one point to the most important factor, and zero for the least important. This was done in order to align the designers, and get an in-depth relation to whats important when measuring the 'goodness' of a concept.

This concludes the first iteration of Discover and Define. In the next section, the second iteration of Discover and Develop will be introduced.

\subsubsection{2. Iteration}

Throughout the second iteration of Discover and Define, the following methods were applied:
\begin{itemize}
    \item Contextual interviews
    \item Paper-prototypes
    \item KJ Technique
    \item Morphology
    \item WOM (Weighted Objectives Matrix)
    \item Personas
\end{itemize}
An update were applied to the following methods outputs from the first iteration:
\begin{itemize}
    \item Proto-personas
    \item Actor Worlds
\end{itemize}

This following section will describe the application of each method and its outcomes.

\begin{itemize}
    \item \bf{Contextual Interviews:}
\end{itemize}

Contextual interviews were conducted on a total of n=10 subjects in the second iteration, where nine of them were classified as job-seekers, and one of them was classified as an 'employer' corresponding with the definitions of the different types of stakeholders, from the proto-personas. The material for the contextual interviews were created in two versions, one for the 'employers' and one for the 'job-seekers'. These two versions were both split into two parts, firstly a structured pre-interview and secondly the actual semi-structured contextual interview. The pre-interview on the job-seeker side, consisted of 27 questions in total, making us able to compare the subjects with the knowledge from existing research, and how well the subjects fit into our proto-personas. The pre-interview on the employer side, were to determine their hiring history, volume and method, in order to give us a nice bedrock for determining how we produce value for these stakeholders in our design. The employer was also asked to prioritize a set of qualities which they are presented with in the pre-interview. These qualities are basic UI-qualities of a 'good' system. The employer also had the possibility of adding qualities to the list, if they felt what they defined as a 'good' system, could not be encapsulated in the existing qualities. This prioritization will be fed into the WOM, so that the measuring of 'goodness' is 'goodness' in the eyes of the employer. Both the contextual interview parts for both the employer and job-seeker walks through the paper-prototypes and asks semi-structured questions in order to assess the subjects reaction to the prototype. Examples of interview structure on the employer side and the job-seeker side can be seen in appendix C [\ref{C}] and appendix D [\ref{D}].

\newpage
\begin{itemize}
    \item \bf{Paper-Prototypes:}
\end{itemize}

The 'Paper-prototypes'-method was utilized as a border-object in combination with the contextual interview, with the stakeholders. The goal of the method was to allow for the stakeholders to successfully communicate their desires for the app, through engaging with the prototype, thereby creating a higher number of tangible suggestions.

\begin{itemize}
    \item \bf{KJ-technique}
\end{itemize}

The KJ-technique was applied in order to take the data from both the co-creation workshop, and the contextual interviews, and prepare it for the morphology. Most data from the discover-methods were ready for implementing in the morphology (data defined as ready is proposed features based on a stakeholder-needs(tangibles). Data defined as not ready is feelings both positive and negative in terms of certain areas of the prototype, which is hard to implement and needs work in order to be a proposed feature(intangibles)). Small individual brainstorms were conducted in order to translate the data in its context to feature propositions. We used the same framework from the first iteration of KJ-Technique, since it still applied.

\begin{itemize}
    \item \bf{Morphology}
\end{itemize}

A morphology was performed in a similar manner as in the first iteration, since the insights given in the second iteration of the discover-phase, did not suggest a larger overhaul of the structure of the prototypes, so the structure defined from the first morphology could be preserved. New suggestions for features were introduced, and built on top of the existing features defined from the morphology in the first iteration. We went with this 'additive'-approach since none of the features were deemed irrelevant in the first iteration of testing.

\begin{itemize}
    \item \bf{WOM}
\end{itemize}

The weighted objectives matrix (WOM) was applied in mostly the same manner as in the first iteration with the same goal. The only difference was with the new concepts, and that we had asked the employer, to rank certain UX-qualities, according to what he deems most valuable. Besides the ranking we also allowed for the employer to propose certain 'features' which he would like the new concepts to be evaluated by in the 'WOM'. The goal with this was to allow the employer not only to be a part of the discover-phase, but also to be a part of the evaluation in the define-phase, and let the employers priorities shine through in the 'WOM'.

\begin{itemize}
    \item \bf{Personas:}
\end{itemize}

The 'Personas'-method was done in order to create ideas around which stakeholders we need to design for. The process of doing proto-personas in the first iteration, relied solely on literature and prior knowledge. Therefore the goal of this method was to integrate the data that we had from the real world, into our proto-personas. The data that went into this process was the data from the contextual interviews of the second iteration, along with the output from a workshop, facilitated by an ITU-group, where the data was sourced from a previous job-center employee, which had experience with refugees and immigrants.

\begin{itemize}
    \item \bf{Proto-Personas:}
\end{itemize}

Based on the first iteration findings from the interviews, the proto-personas were updated to represent both the studied population and the population represented in the literature for both types of actors.

\begin{itemize}
    \item \bf{Actor Worlds:}
\end{itemize}

Based on the knowledge from the first iteration, the 'Actor-worlds'-output was updated to convey the newly gained knowledge about relations between actors. The essence of the update was to highlight the relation between 'The Seed' and the category 'samarbejdspartner'. This relation was discovered since we got aware of the cooperation between The Seed, government institutions, and NGO's, in the development of services and partnerships which justifies the new overlapping relationship between the two actor worlds.

\subsubsection{3. Iteration}

Throughout the third iteration of Discover and Define, the following methods were applied:
\begin{itemize}
    \item Contextual interviews
    \item Paper-prototypes
    \item Usability testing
\end{itemize}

The methods application and goals, will be detailed in the section below.

\begin{itemize}
    \item \bf{Contextual interviews}
\end{itemize}

The contextual interviews were conducted similarly to previous iterations. This time the method worked in conjunction with the paper-prototype-method and the Usability-testing-method, both for employers and job-seekers. The goal of the method was to produce, tangible statements from both groups of users, which could be implemented as features in the next iteration of Develop and Deliver. The reason for this was that the deadline for final-prototype submission approached, so some of the usual processing of more intangible data, was not possible. Another goal of applying this method along with the usability-testing-method was to allow the interviewer to focus on outliers in performance, by asking semi-structured questions directly to the tester, as the outlier is produced. This allows the Usability-testing-method to focus on describing general trends in the analysis-part. 

\begin{itemize}
    \item \bf{Paper-Prototypes:}
\end{itemize}

The 'Paper-prototypes'-method was utilized as a border-object in combination with the contextual interview, just as the previous iteration. The stakeholders were asked to interact with the prototype, where they previously were just asked to reflect on the visual aspects and function descriptions we gave them. They were asked to use their finger, to navigate through the app, solving a number of tasks given by us as moderators.

\begin{itemize}
    \item \bf{Usability testing}
\end{itemize}

The 'Usability testing'-method, was applied with inspiration from David Benyons 'Designing User Experience'\cite{benyon2019designing}. This framework proposed a way of testing for usability in a structured manner. Based on the latest click-through-prototype, we produced questions which we would like to be answered which were:

\begin{itemize}
    \item Is the system intuitive?
    \item Is there any difference in performance between refugees and ethnic Danes?
    \item Can people access the information they need in an optimal and correct manner?
\end{itemize}

These questions led to preparing a number of tasks for testing, both for the 'job seeker'-prototype and the 'employer'-prototype. These tasks would guide the testers through numerous levels of functionalities in the prototype. As part of the preparation, an interview guide was conducted, as well as an optimal path for task-completion, from which deviance from this path, would not let the tester progress to the next step or page, and be recorded as an error. We recorded four key measures as the tests were conducted:

\begin{itemize}
    \item Total time for task completion
    \item Number of wrong clicks compared to the 'optimal'-path
    \item Number of question regarding their lack of understanding in solving the task, or using features
    \item Comprehension - How intuitively they accessed information within the prototype
\end{itemize}

The tasks were created with difficult levels of complexity, where the ranking from easiest to hardest for job-seekers are:

\begin{itemize}
    \item Task 1
    \item Task 2
    \item Task 3
    \item Task 4
\end{itemize}

Where for employers the task ranked from easiest to hardest were:

\begin{itemize}
    \item Task 2
    \item Task 3
    \item Task 1
\end{itemize}

The method was conducted in combination with the Paper-prototypes method, and the contextual interview method. The goal of the tests, were to effectively answer the questions stated above for both sets of prototypes and actors, and try to generalize the results from the tests to the whole stakeholder group. The goal of combining the three methods was to get more tangible data through utilizing a border object in the paper-prototype, and be able to ask follow-up questions in regards to testers performance in the test, through the contextual interview, looking for causality.


\subsection{Develop \& Deliver}
The Develop \& Deliver phases includes the following methods used throughout the iterations when needed[\ref{DevelopandDeliver}]: 

\begin{figure}[H]
\caption{Methods used in the Develop and Deliver phase}
\centering
\label{DevelopandDeliver}
\includegraphics[width=250pt]{Images/Methods/Develop and deliver.PNG}
\end{figure}

In the following section, iterations 1 through 3 will be presented, as in the previous section of Discover And Define.

\newpage
\subsubsection{1. Iteration}

Throughout the first iteration of the Develop- and Deliver-phases, the following methods were applied:
\begin{itemize}
    \item Storyboarding
    \item Paper-prototypes
    \item Role-Playing
    \item Click-Through-Prototype
\end{itemize}

The methods application and goals will be covered in the section below.

\begin{itemize}
    \item \bf{Storyboarding}
\end{itemize}

Storyboarding was done after the final concept had been chosen, in order to fully visualize a flow of the chosen concept. Each of us created a storyboard, one based on the Wall Walk structure, and one based on the structure from the Morphology (which by extension is a structure from the Function Means Tree). We detailed both structures in plenum before doing the storyboard, in order to ensure full alignment between the designers. As the storyboards were produced, we compared them, and with the new knowledge in mind, we composed a final flow which were to be the 'feature-list' from which the first prototype should be conceived from. 

\begin{itemize}
    \item \bf{Click-Through-Prototype}
\end{itemize}

The 'Click-Through-Prototype' was developed based one the first iteration of the Discover and Define stages, in the application 'Just-in-Mind'. The goal of developing the click-through-prototypes, was to have a tangible border-object which allowed for further data-collection with our stakeholders. A separate goal of producing the prototype is the learnings you gain when you have to manifest your ideas in something real, which sometimes raises issues and insights, that you had not thought of. Just-in-Mind is PC-based application which allows the user to design an interactive prototype of both a smartphone-app and a desktop-app.

\begin{itemize}
    \item \bf{Role Playing}
\end{itemize}

The 'Role-playing'-Method was applied when the prototypes were developed, with the goal of simulating the response of our stakeholders, to the newly developed prototypes, and to assess the time it will take to go through a full contextual interview on both the 'employer'-side and the 'job-seeker'-side. The designers were each assigned a task: One was the tester, and one was to play the role of the stakeholder (either employer or job-seeker). The tester would run through a simulated test with authentic questions, and the user would try to take the perspective of the stakeholder we were testing for, in order to give an indication of whether the test had any loop-holes which needed to be addressed. The time-perspective was in order to have some sort of estimate of total interview length, so we would know how much time we should ask of the stakeholders in our stakeholder recruitment.

\begin{itemize}
    \item \bf{Paper-prototypes}
\end{itemize}

The 'Paper-prototype'-method was applied indirectly since the development of the first prototype was done on the 'Just-in-mind'-app. The different pages of the app was then printed and interacted with by the stakeholders, in a chosen order which suited the contextual interview which the paper-prototypes were introduced in. It was therefore a prototype in the Deliver and Develop Phase, but will be utilized as a border-object in the 2. and 3. iteration of Discover and Define.

\subsubsection{2. Iteration}

Throughout the second iteration of Develop and Deliver, the following methods were applied:

\begin{itemize}
    \item Co-Creation workshop
\end{itemize}

The following outputs from previous methods were updated in this iteration:

\begin{itemize}
    \item Click-Through Prototype
\end{itemize}

In the following section the methods applications and updates will be described, as well as the outcome of the methods.

\begin{itemize}
    \item \bf{Co-creation workshop}
\end{itemize}

The Co-creation workshop was conducted with n=4 'job-seekers' who had completed the pre-interview prior to attending the workshop. The results from the pre-interviews were processed prior to the workshop, and was the bedrock of the co-creation workshop. The co-creation workshop lasted one and a half hour, and was split in two parts: A discussion part, and a creation part. The first part was to discuss some negative feelings or emotions that the stakeholders had expressed in the pre-interview about the prototype, which they in pairs, were asked to discuss and formulate through drawing or verbally how the feeling could be avoided by changing the prototype. The second part focused on different parts of the prototype, where they had expressed different ways to implement functionality then it was done by us, the designers. The suggestions were compiled into a slideshow, and presented to them in a visual manner, and they were asked to ideate further and present their ideas in pairs. The last part of the workshop consisted of a session were we went through different priorities in terms of what content they would like to be available on the platform prototype, to prepare them properly for the job-process. They were each individually asked to rank different types of content and types of preparation material for job-interviews, in order to highlight what provides most value for the job-seekers.

\begin{itemize}
    \item \bf{Click-Through prototype}
\end{itemize}

The Click-Through-Prototypes were updated in accordance with the selected concept from the WOM of the second iteration of Discover and Define.

\subsubsection{3. Iteration}
\label{Develop and Deliver-3}

Throughout the third iteration of Develop and Deliver, the following methods were updated:

\begin{itemize}
    \item Click-Through Prototype
\end{itemize}

A description of the update of the method, will be presented below.

\begin{itemize}
    \item \bf{Click-Through prototype}
\end{itemize}

The Click-through prototype was updated based on the input from the three methods; Contextual interview, Usability testing, and paper-prototypes. The data produced from these three methods were a list of tangible features or updates of the app, which provided value for the stakeholders which were subjects for testing in the third iteration.

analysis and results: 

The results produced from the methods will be presented below, in a similar structure to the way the methods-section was presented. Along with the outcomes, an overall analysis will be presented as a diamond is finalised (a diamond consists of either a Discover- and Define-phase or a Develop- and Deliver-phase). This analysis seeks to interpret the results of the methods, and present either what has been validated or verified. 

\subsection{Discover And Define}

\subsubsection{1. Iteration}

Throughout the first iteration of Discover And Define, the following methods were applied:

\begin{itemize}
    \item Proto-Personas
    \item PACT
    \item Actor worlds
    \item Contextual interview
    \item Function Means Tree
    \item Morphology
    \item KJ Technique 
    \item Wall Walk
    \item User Specification
    \item WOM (Weighted Objectives Matrix)
\end{itemize}

Each methods outcome will be detailed in the section below.

\begin{itemize}
    \item \bf{Proto-Personas:}
\end{itemize}

The method resulted in a total of four proto-personas, illustrated in the the two figures below:[\ref{Proto1_1}] and [\ref{Proto1_2}]. The four personas, guided us in terms of stakeholder-acquisition, and in talks with The Seed, we used it as a basis for evaluation of stakeholders as they were recruited, deciding whether or not the recruited lived up to our personas, and if we needed to acquire additional stakeholders.

\begin{figure}[H]
\caption{First set of Proto-Personas produced in the first iteration}
\centering
\label{Proto1_1}
\includegraphics[width=\textwidth]{Images/Methods/Iteration1/proto1_1.PNG}
\end{figure}

\begin{figure}[H]
\caption{Second set of Proto-Personas produced in the first iteration}
\centering
\label{Proto1_2}
\includegraphics[width=\textwidth]{Images/Methods/Iteration1/proto1_2.PNG}
\end{figure}

\begin{itemize}
    \item \bf{PACT:}
\end{itemize}

The outcome from the PACT-method can be seen in figures below: First two columns:[\ref{PACT1_1}] and the last two columns: [\ref{PACT1_2}]. The PACT-model guided us in determining the initial actions which a job-seeker and an employer would go through in a job-seeking process. This knowledge had an impact on our prototype, since it established some initial ideas of what actions the prototype should support, and the framework forced us to consider disabilities, and how the prototype should be designed to also be functional for people with disabilities. 

\begin{figure}[H]
\caption{First two PACT-columns produced in first iteration}
\centering
\label{PACT1_1}
\includegraphics[width=\textwidth]{Images/Methods/Iteration1/Pact1_1.PNG}
\end{figure}

\begin{figure}[H]
\caption{Last two PACT-columns produced in first iteration}
\centering
\label{PACT1_2}
\includegraphics[width=\textwidth]{Images/Methods/Iteration1/Pact1_2.PNG}
\end{figure}

\begin{itemize}
    \item \bf{Actor Worlds:}
\end{itemize}

The output of the 'Actor Worlds'-method can be seen in the figure[\ref{Aktørverdener}].

\begin{figure}[H]
\caption{Actor Worlds defined from the first iterations}
\centering
\label{Aktørverdener}
\includegraphics[width=\textwidth]{Images/Methods/Iteration1/Aktørverdener.PNG}
\end{figure}

The method allowed us to get a broader view of the direct stakeholders connected, with the product, and allowed us to effectively select relevant stakeholders to take into account when designing, based on the closeness of relations.

\begin{itemize}
    \item \bf{Contextual interviews:}
\end{itemize}

The outcome of the contextual interviews were intangibles and values, which were translated to post-its with statements like 'I, the employer, would like *insert value that a good system has*'. These post-its will be fed into the 'KJ-Technique'-method described in the subsequent subsection. Along with the intangibles, concrete feature propositions known as 'tangibles' were also produced. 

\begin{itemize}
    \item \bf{Function Means Tree:}
\end{itemize}

The outcome of the 'Function Means Tree'-method is illustrated in the figure below [\ref{Tree}].

\begin{figure}[H]
\caption{Function Means Tree with the transferred layers for the morphology in green}
\centering
\label{Tree}
\includegraphics[width=\textwidth]{Images/Methods/Iteration1/Function Means Tree.PNG}
\end{figure}

As highlighted on the figure, the branches marked in green, will be the columns of the Morpho-logy matrix, since we deemed it a sensible level to ideate at. Along with the columns for the morphology, a small brainstorm was applied to each branch effectively adding ideas for means, to perform each function (functions marked in green, means in boxes under branches).

\begin{itemize}
    \item \bf{Morphology:}
\end{itemize}

The outcome was five different concept descriptions which would go on to be evaluated in the WOM-method. The five concepts were:

\begin{itemize}
    \item Minimum Bias
    \item Maximum Functionality
    \item Minimal Involvement from The Seed
    \item Minimum Viable Product
    \item Optimum Viable Product
\end{itemize}

Along with the titles of the concepts, a coded description of which functions the concept would have was also created, based on the cell-number of the morphology. The Morphology can be seen in Appendix B[\ref{B}].

\begin{itemize}
    \item \bf{Wall Walk:}
\end{itemize}

The resulting board of the Wall Walk is illustrated in the following figure[\ref{Wall}]. 

\begin{figure}[H]
\caption{Wall Walk based on the structured output of the KJ-technique}
\centering
\label{Wall}
\includegraphics[width=400pt]{Images/Methods/Iteration1/Wall Walk.jpg}
\end{figure}

Out of the suggested Solutions and Problems, we went through them, and selected the ones, which we found vital for the first round of prototyping, which would classify them as 'hot ideas'. This resulted in a list of solutions and problems where the solutions went straight into the morphology matrix as ways to implement or improve a certain function. The list of problems were subject of another brainstorm, addressing these problems by proposing solutions to these and putting them into the morphology matrix.

\newpage

\begin{itemize}
    \item \bf{KJ-Technique:}
\end{itemize}

As the post-its were placed in silence, titles for the different groups were given as shown in the resulting figure[\ref{KJ}].

\begin{figure}[H]
\caption{KJ-technique - Initial categories with input from the 'employer' and the 'job-seeker'}
\centering
\label{KJ}
\includegraphics[width=\textwidth]{Images/Methods/Iteration1/KJ technique.jpg}
\end{figure}

We, the designers, then decided to change the structure, in which the data was presented in, so that it resembles a flow of steps, as seen in the figure[\ref{KJ2}]

\begin{figure}[H]
\caption{KJ-technique - Subsequent categories with input from the 'employer' and the 'job-seeker'}
\centering
\label{KJ2}
\includegraphics[width=400pt]{Images/Methods/Iteration1/KJ technique 2.jpg}
\end{figure}

This was to better see how the feedback from the different stakeholders would infer on the other steps in the UI. Arrows would be drawn to simulate our UI from beginning to end. This structure was the ending point of the 'KJ-technique'-method, and is the initial step of the 'Wall-Walk'-method.

\begin{itemize}
    \item \bf{User Specification:}
\end{itemize}

The User specification can be found in the Appendix A [\ref{A}].

\begin{itemize}
    \item \bf{WOM (Weighted Objectives Matrix):}
\end{itemize}

 The Pugh-matrix which decided the weight of our factors can be seen in table[\ref{Pugh}] below. Our subsequent WOM can be seen in figure[\ref{WOM}]below. The WOM-method stated that the 'OVP' which is called the 'Optimum Viable Product', was the best of the concepts, and we therefore moved on to storyboard this concept, based on the output from the method which was the selection framework and the selection.
% Please add the following required packages to your document preamble:
% \usepackage{longtable}
% Note: It may be necessary to compile the document several times to get a multi-page table to line up properly
\begin{longtable}{|>{\columncolor[gray]{0.9}}l|>{\columncolor[gray]{0.9}}l|l|l|l|l|l|l|l|}
\caption{Pugh-matrix ranking factors between each other}
\label{Pugh}\\
\hline
\rowcolor[gray]{0.9}
\textbf{Factors weight}  &  & a & b & c & d & e \\ \hline
\endhead
%
\textbf{Flow} & a & N/A & 1 & 0 & 0 & 0 \\ \hline
\textbf{Tilgængelighed} & b & 0 & N/A & 0 & 0 & 0 \\ \hline
\textbf{Troværdighed} & c & 1 & 1 & N/A & 0 & 0 \\ \hline
\textbf{Tidsforbrug} & d & 1 & 1 & 1 & N/A & 0 \\ \hline
\textbf{Udviklingsgrad} & e & 1 & 1 & 1 & 1 & N/A \\ \hline
SUM &  & 3 & 4 & 2 & 1 & 0 \\ \hline
\end{longtable}

% Please add the following required packages to your document preamble:
% \usepackage{longtable}
% Note: It may be necessary to compile the document several times to get a multi-page table to line up properly
\begin{longtable}{|>{\columncolor[gray]{0.9}}l|l|l|l|l|l|l|l|l|l|l|l|}
\caption{WOM scoring and weighting of morphed concepts}
\label{WOM}\\
\hline
\rowcolor[gray]{0.9}
 & \multicolumn{1}{c|}{W.} & \multicolumn{2}{c|}{Min. Bias} & \multicolumn{2}{c|}{Max. func.} & \multicolumn{2}{c|}{Min. Involv.} & \multicolumn{2}{c|}{MVP} & \multicolumn{2}{c|}{OVP} \\ \hline
\endhead
%
\rowcolor[gray]{0.9}
Req. & * & Score & Value & S & V & S & V & S & V & S & V \\ \hline
Flow & 3 & 2,5 & 7,5 & 1,5 & 4,5 & 3 & 9 & 4 & 12 & 4 & 12 \\ \hline
Tilgængelighed & 4 & 2 & 8 & 4,5 & 18 & 2 & 8 & 2,5 & 10 & 4,5 & 18 \\ \hline
Troværdighed & 2 & 3,5 & 7 & 4,5 & 9 & 1,5 & 3 & 4 & 8 & 4,5 & 9 \\ \hline
Tidsforbrug & 1 & 2 & 2 & 2 & 2 & 1,5 & 1,5 & 4 & 4 & 3,5 & 3,5 \\ \hline
Udvikling & 0,5 & 4 & 2 & 1 & 0,5 & 5 & 2,5 & 3,5 & 1,75 & 2,5 & 1,25 \\ \hline
\textbf{Total value} &  &  & 26,5 &  & 34 &  & 24 &  & 35,75 &  & 43,75 \\ \hline
\end{longtable}

This marks the end of the first iteration of Discover and Define. In the following subsection, an overall analysis of the results will be presented, based on the output from the methods described in the now finished section.

\subsubsection{Overall Analysis: Discover And Define 1. iteration}

Within the first iteration of Discover and Define, the input material for the methods just described, was mainly impressions from literature, The Seed, and our personal knowledge and assumptions regarding the stakeholders in question. The three methods allowing us to formalize that input was PACT, Proto-personas, and Actor Worlds, Where we analysed the stakeholders themselves, their actions, their access to technology, and the arena we are designing in with The Seed as the root of the arena.\\

\newpage

Since the three methods were made within quick succession, little knowledge was added in between using the methods. That is mainly the reason for the portrayal of the stakeholders in the PACT-analysis and the Proto-personas is mainly the same. The Framework of PACT forces you to consider you stakeholder in question, as a part of the general population, as opposed to the proto-personas which focused a lot on the unique properties of the stakeholder. That led to realising the general disabilities (i.e color-blindness, arthritis, bad sight etc.) also applied to both the stakeholder groups, where the proto-personas did not facilitate this learning. This suggests an inclusive design approach, taking into account disabilities, different access to technologies, and the vastness of activities which the system potentially can support, since we at that point did not have any empirical knowledge of how the process worked in the real world.\\

The Actor worlds method made us realize that the context which The Seed, as an actor, itself resides in, will pose a potential trade off scenario. The Seed has a clear mission of changing the recruitment process in numerous way, and in their own words create 'disruption'. This entails fundamental changes to what information is asked for when profiling the stakeholders in theirs system, and also what information is provided about the stakeholders to each other. This information can be perceived as valuable by the stakeholders within the system, whether or not it really provides value, or is just anecdotal. This desire for revolution from The Seeds side, might compromise with perceived user-value, and therefore needs to be addressed. We need to design a system which that removes or limits information that can lead to bias, but where do we draw the line, so we provide enough user-perceived value, and still satisfy The Seed's request for disruption? This goes to say that The Seed's request is an extension of their belief-system which says that bias needs to be removed, and is therefore fair to make, but we foresaw the need for a trade-off at that point. 
Another smaller realization from the Actor World-method, was the importance of interaction and communication between the job center and the job-seeker. There are certain requirements by law, which the job-seeker have to meet, in order to receive their government support. This is i.e a certain number of times, where you as a job-seeker show an interest in a job. Since we now digitize the interaction between the job-seeker and the employer, this interaction needs to be documented, and comply with government standards. This is although blackboxed during this project, due to our main concern being the job-seeker and employers.\\

The employer was analysed to be rather different in the proto-personas method, compared to the contextual interview. In the contextual interview the employer was patient and thorough in the recruitment process, where the proto-persona was the opposite, being impatient and unwilling spend a lot of time on recruitment. The difference led to the realisation that the UI for the employer needed to allow for both approaches to co-exist, a simplistic and fast approach, and a more thorough and complexity-hungering approach. These approaches, if not taking into account, might cancel each other out, since the one might be turned off by UI being too cumbersome if too much complexity is introduced, and the other lacking trust in the process, since less information is presented than usual. \\

The insights from the Discover-phase, presented us with a diverse set of ill-defined problems, which needed to be defined, and that is were the second level of the diamond comes into play, as we seek to converge towards a tangible problem definition. \\

The Define-phase was comprised of a total of 6 methods, seeking to diverge onto a more tangible problem definition, from the intangible data and impressions provided from the Discover-phase.\\

With the goal of producing tangible features, we utilized two different methods, to see if different approaches to going from intangible to tangible features would yield different results. A model of the realized Define-Phase can be see in figure:[\ref{DefineReal}].

\begin{figure}[H]
\caption{Realized Define-phase 1. Iteration (output=box, Method=Circle)}
\centering
\label{DefineReal}
\includegraphics[width=\textwidth]{Images/Analysis & results/Define-phase realized.PNG}
\end{figure}

The structuring of data done in the KJ technique and the Function Means Tree, differs in the sense that one is a bottom-up-approach, where the other is a top-down-approach. The KJ technique which is the bottom-up-method, took empirical observations and grouped them together as they thematically was perceived, by the designers, to belong together. This yielded a flow-like-structure where the three stakeholders:'The Job-seeker', 'The Employer', and 'The system' interacted in a chronological way. The other approach from the Function Means Tree-method, the top-down-method, approached it differently. This produced a non-linear structure, rooted in a single primary function, and then proposing means to provide this function. The value Provided by doing both these methods became apparent when we proceeded to do ideation within both frameworks. The methods created different spaces for ideation, which prompted us to propose different features, depending on which space was ideated in. The Process and output of utilizing the two approaches is illustrated in the figure:[\ref{Func_KJ}]. 

\begin{figure}[H]
\caption{Data-structuring: Top-down vs. bottom-up approach}
\centering
\label{Func_KJ}
\includegraphics[width=\textwidth]{Images/Analysis & results/Func_KJ.PNG}
\end{figure}

The ideation is the process of going from the UI-spaces, to the feature space.\\

The process also framed our design to approach the UI both in a general sense, and a concrete sense. Where features in a general sense would transcend the UI structure, and influence decisions in many steps of the UI. Where the concrete steps would infer on more specific points of the UI. The methods produce two different perspectives of the stakeholder 'The system'. One defines the stakeholder based on what the stakeholder does, one defines the stakeholder on what it should do.\\

From the literature review, and conversation with The Seed, we had an idea of which information can lead to bias. The factors which commonly produce unfair bias are:

\begin{itemize}
    \item Age
    \item Gender
    \item Cultural Background
    \item Language abilities
    \item Writing abilities
    \item Economical status
    \item Religion
    \item Digital Illiteracy
    \item Social capital
\end{itemize}

From these bias factors, we went through the UI-shaped structure produced by the KJ Technique and identified potential sources of information, which could lead to bias, producing the figure beneath:[\ref{Bias}].

\begin{figure}[H]
\caption{Information sources from UI-structure which can cause bias}
\centering
\label{Bias}
\includegraphics[width=350pt]{Images/Analysis & results/Bias_sources.PNG}
\end{figure}

Some bias-factors are not affected by the UI, namely Economic Status, Digital Illiteracy and Social Capital. This is due to the scope of our project, and that the UI facilitates an inter-personal process, where implicit bias factors are most predominant. If we look at the bias factors not affected by the UI, they are more influential on the systemic level. Lack of social capital hinders the possibility of getting a job by reference through network which is systemic bias, but on the implicit inter-personal level, an employer would not reject an applicant due to having low social capital. So, the factors remaining does not apply in the realm we are aiming to design for, and will therefore not be treated at any point of the design stages, since it is out of scope.\\

Therefore, we come to the realization that these sources of information in figure[\ref{Bias}], needs to be either removed or minimized, in the Develop and Deliver-phase, since they are contributing to bias. What information needs to be removed, and what information needs to be minimized, will be determined by the before mentioned user-value and disruption trade-off, balancing what information the user sees as valuable and what The Seed envisions.\\

In terms of deviating from prescribed method-usage, We deviated from the traditional way of using the User Specification-method, by not using it as a control measure for a final evaluation as a prototype is produced. This is due to the fact, that the User Specification we produced, contained a lot of demands, which only a full-functioning system could live up to. Since we knew that we would only be able to produce a semi-functioning prototype, we deselected using the framework for evaluating the final prototype, and chose to only let the criteria infer on the decision parameters of the WOM. 
We deviated from the traditional way of selecting parameters in the WOM. We added a parameter which was not part of the User Specification which was 'udviklingsgrad'. This parameter was added for project-timeline specific reasons, since we had a deadline for producing a prototype, and therefore the parameter would reward concepts which were less cumbersome to develop.\\

The main points of the analysis above is presented in the bullets below, and will be dealt with in the overall analysis section of the Develop and Deliver-phase 1. iteration:

\begin{itemize}
    \item Design for inclusivity
    \item User-oriented design vs. Disruption
    \item Simplicity vs. Complexity
    \item Designing UI on a concrete and general level
    \item Removing or minimizing Bias-enforcing information sources
\end{itemize}

These bullets are the overall problem definition bullets, where sub-problems are described in the in the method-specific results-sections as well. This concludes the overall analysis, and the 2. iteration of Discover And Define will be detailed in the next section. 

\subsubsection{2. Iteration}

Throughout the second iteration of Discover and Define, the following methods were applied:
\begin{itemize}
    \item Contextual interviews
    \item Paper-prototypes
    \item KJ Technique
    \item Morphology
    \item WOM (Weighted Objectives Matrix)
    \item Personas
\end{itemize}

Updates were applied to the following methods outputs from the first iteration:

\begin{itemize}
    \item Proto-personas
    \item Actor Worlds
\end{itemize}

This following section will describe the results of each method used in the second iteration.

\begin{itemize}
    \item \bf{Contextual Interviews:}
\end{itemize}

The result of the contextual interviews were a set of impressions from the stakeholders, which could be either tangible or intangible. The intangible (feelings, emotions towards the prototype) was subject for further processing, going through both the KJ-technique and the Wall Walk in order to end out as tangible features which could be implemented in the newest prototype. The Tangibles were added directly to the Morphology matrix, since they were ready for concept-drawing. All the processed intangibles and the tangibles can be seen in appendix E, as the cells contents within the morphology: [\ref{E}].

\begin{itemize}
    \item \bf{Paper-prototypes:}
\end{itemize}

Outcome of the paper-prototypes, used as a border object, was that it allowed for the interviewed subjects to come up with a higher number of tangible ideas and expressions of stakeholder value. As stakeholders had a hard time elaborating on their feelings in general, the border-object was addressed, and the interviewer tried to make the stakeholder communicate their emotion through suggesting changes to the border-object which would negate a negative emotion, or enhance a positive emotion.

\begin{itemize}
    \item \bf{KJ-Technique:}
\end{itemize}

 The outcome of the KJ-Technique was a set of propositions for features, which were based on the intangible data from the stakeholders. These feature propositions were then structured to fit into the morphology columns, and marks the starting point of the Morphology-method.

\begin{itemize}
    \item \bf{Morphology:}
\end{itemize}

An overview of the morphology performed in the second iteration can be seen in appendix E: [\ref{E}]. The outcome of the morphology was 5 concepts and a coded description of what functionality the concepts would have, based on the cell-number of the morphology matrix. The 5 concepts chosen for the WOM was:

\begin{itemize}
    \item Prototype 1
    \item Maximum Functionality
    \item Best for Applicant
    \item Best for Employer
    \item True to The Seed
\end{itemize}

\newpage

\begin{itemize}
    \item \bf{WOM (Weighted Objectives Matrix):}
\end{itemize}

The WOM can be seen in the table: [\ref{WOM2}]. The outcome of the method was the selection of the concept 'True to The Seed' as the best concept, as well as the framework it self.

% Please add the following required packages to your document preamble:
% \usepackage{longtable}
% Note: It may be necessary to compile the document several times to get a multi-page table to line up properly
\begin{longtable}{|>{\columncolor[gray]{0.9}}l|l|l|l|l|l|l|l|l|l|l|l|}
\caption{WOM second iteration}
\label{WOM2}\\
\hline
\rowcolor[gray]{0.9}
 & \textbf{W} & \multicolumn{2}{c|}{\textbf{Proto 1}} & \multicolumn{2}{c|}{\textbf{Max. Func.}} & \multicolumn{2}{c|}{\textbf{Best Appli.}} & \multicolumn{2}{c|}{\textbf{Best Emplo.}} & \multicolumn{2}{c|}{\textbf{True Th. Seed}} \\ \hline
\endhead
%
\rowcolor[gray]{0.9}
Requirements & * & S & V & S & V & S & V & S & V & S & V \\ \hline
Flow & 3 & 2,5 & 7,5 & 3 & 9 & 3 & 9 & 3,5 & 10,5 & 4,5 & 13,5 \\ \hline
Tilgængelighed & 4 & 2,5 & 10 & 3,5 & 14 & 4 & 16 & 4,5 & 18 & 4,5 & 18 \\ \hline
Troværdighed & 2 & 2 & 4 & 4,5 & 9 & 3 & 6 & 3 & 6 & 3,5 & 7 \\ \hline
Tidsforbrug & 1 & 3,5 & 3,5 & 2,5 & 2,5 & 2 & 2 & 4 & 4 & 4 & 4 \\ \hline
Udviklingsgrad & 0,5 & 5 & 2,5 & 1,5 & 0,75 & 2,5 & 1,25 & 3,5 & 1,75 & 3 & 1,5 \\ \hline
Indsigt & 0,5 & 1 & 0,5 & 5 & 2,5 & 3,5 & 1,75 & 3 & 1,5 & 3 & 1,5 \\ \hline
Relation & 0,5 & 1 & 0,5 & 4,5 & 2,25 & 3 & 1,5 & 2 & 1 & 4 & 2 \\ \hline
Total value &  &  & 28,5 &  & 40 &  & 37,5 &  & 42,75 &  & 47,5 \\ \hline
\end{longtable}

\newpage

\begin{itemize}
    \item \bf{Proto-Personas and Personas:}
\end{itemize}

The resulting update can be seen in the following figures: The first two personas:[\ref{Proto2_1}] and the last two personas: [\ref{Proto2_2}].

\begin{figure}[H]
\caption{First two Proto-Personas updated for second iteration}
\centering
\label{Proto2_1}
\includegraphics[width=\textwidth]{Images/Methods/Iteration2/Proto2_1.PNG}
\end{figure}

\begin{figure}[H]
\caption{Last two Proto-Personas updated for second iteration}
\centering
\label{Proto2_2}
\includegraphics[width=\textwidth]{Images/Methods/Iteration2/Proto2_2.PNG}
\end{figure}

The differences in the new update compared to the old one, will be covered in the overall analysis.

\begin{itemize}
    \item \bf{Actor Worlds:}
\end{itemize}

The output of the update of the 'Actor Worlds'-method can be seen in the figure[\ref{Aktørverdener2}].

\begin{figure}[H]
\caption{Actor Worlds updated for the second iteration}
\centering
\label{Aktørverdener2}
\includegraphics[width=\textwidth]{Images/Methods/Iteration2/Aktørverdener 2.PNG}
\end{figure}

\subsubsection{Overall Analysis: Discover And Define 2. Iteration}

The update given to the proto-personas from the first iteration, was applied with the goal of representing the interviewed stakeholders from the contextual interviews. four of the total nine stakeholders, which classified as job-seekers, also qualified as refugees or immigrants, so they as well as the employer was the input for the update. The employer-personas qualities in the updated version of the personas[\ref{Proto2}], is now a mix of the assumption we as designers have, based on prior knowledge and literature, and the previous version of the employer-persona[\ref{Proto}]. Where the employer persona previously were rather one-dimensional, it became apparent that a two-dimensional spectrum of stakeholder action was existing. The thoroughness of the interviewed employer was very apparent, and places the new stakeholder on a different place on the axis in the figure below:[\ref{persona_update}].

\begin{figure}[H]
\caption{Employer Axis: Level of Thoroughness}
\centering
\label{persona_update}
\includegraphics[width=\textwidth]{Images/Analysis & results/Persona_update.PNG}
\end{figure}

\newpage
Since both ends of the axis is presumed to exist on a more general level, design actions need to be made, in order to allow both use-patterns to exist. One is fast paced, superficial, and impatient in its style of recruitment, and the other is slow paced, thorough, and patient.

The first versions of the personas, from the first iteration[\ref{Proto}], is a broad spectrum of the refugees and immigrants described in literature. These personas has a broad range of educational backgrounds, language abilities, religious practices etc. There is although a tendency for the first personas to be on the lower level of all the stated parameters, due to the literature often being concerned with the problems with integration, and therefore often highlighting individuals which have lower levels of education, language abilities and a well established religious practice, because that is where problems with integration come to shine most times. The second iteration of the personas[\ref{Proto2}], positions the job-seeker at the more easily 'integrative' in terms of the parameters previously discussed. This is due to the stakeholder recruitment done by The Seed, focuses on individuals who are 'job-ready', and in general individuals who are very close to, or all ready integrated within the danish labour market system. This is a strategic choice by The Seed, since the go-to-market strategy has identified this group of stakeholders as the most relevant in terms manifesting their technology in a market. If you compare the two perceptions of the general populations of refugees and immigrants, from the two persona iterations, and ask the question, whether they are representative? We are inclined to say no. We believe that they are at opposite ends of the spectrum: 'ease of integration', since literature is mostly concerned with individuals who have a hard time getting integrated, and the interviewed stakeholders seem to have a better chance of getting integrated into the danish labour market system. The design implications of this spectrum, will be that we have to design for both realities. One example is, that we cannot expect all to have the language competence, of the 4 interviewed stakeholders from the contextual interview, so the inclusion of symbolic representations instead of a lot of text, could be preferred, since it is a more inclusive approach. 

We maintained the structure of the methods in the define phase, since the input given fit the all-ready created framework from the first iteration. Due to time limitations, methods were deselected, and a mostly linear process was conducted. We deviated from the standard of doing the WOM, in that we added the parameter 'relation', for determining the 'goodness' of a concept. This parameter was established as important during the co-creation-workshop, since the participants indicated that a high level of relation between stakeholders, facilitated by the prototype, would increase their perceived 'goodness' of the system. 

Throughout discussions with The Seed, we identified the CV as a potentially bias-causing source of information. The main argument was that not due to the systemic bias towards refugees and immigrants, which hinders their overall access to employment, the employment record in the CV is bias-enforcing. The bias-contribution can be extracted into two sub-components, which both cause different levels of bias:

\begin{itemize}
    \item The duration of employment
    \item The 'free text' involved in the employment description
\end{itemize}

\newpage

The Seed is critical towards the duration of employment being informed, due to refugees and immigrants having a harder time accumulating years of relevant experience. This is due to numerous things, since they are often fired earlier than their ethnically danish counterparts, and their access to employment is also at a deficit as it has been established earlier in the report. The 'free text' is the text used to describe ones concrete responsibilities in a position. Due to the established language deficit for refugees and immigrants, this can be a bias-producing factor, creating a deficit for this group, compared to ethnic Danes. As it was established earlier in the iteration, this information is considered valuable by the employer, since they think experience in the sector/job they are recruiting for, translates to a better performance in the position they are seeking. This again calls for a trade-off between employer-value and The Seeds vision for disruption, which will need to be addressed in the Develop and Deliver-phase.

The summary ideas and problems presented in this overall analysis are:

\begin{itemize}
    \item CV trade-off: Employer value vs. The Seed disruption
    \item Thorough employer vs. Impatient employer
    \item Highly integrateable vs. less integrateable
\end{itemize}

This concludes the second iteration of Discover and Define. Below here you will find the third iteration of Discover and Define.

\subsubsection{3. Iteration}

Throughout the third iteration of Discover and Define, the following methods were applied:
\begin{itemize}
    \item Contextual interviews
    \item Paper-prototypes
    \item Usability testing
\end{itemize}

Throughout the section below, the outcomes of the methods will be presented.

\begin{itemize}
    \item \bf{Contextual Interviews:}
\end{itemize}

The outcome was a set of features or updates to the prototype, which went into developing the final click-through prototype without any further processing, besides a discussion between the designers, whether the suggested feature or update was relevant across all stakeholders, or if it was a one-off. This goes to say that all data from this round of contextual interviews was tangible. A separate list was also created, in order to list things, which would not be able to be implemented in time before the deadline.

\begin{itemize}
    \item \bf{Paper-Prototypes:}
\end{itemize}

The outcome was a higher level of tangible data, since the method was utilized along with the contextual interview and the usability testing method. 

\begin{itemize}
    \item \bf{Usability-testing:}
\end{itemize}

The outcome of the Usability testing was a set of test-scores, which have been processed into key figures for further analysis. The table of results from the job-seeker-stakeholders can be seen beneath here:\ref{UsabilityJob}. The table of results from the employer-stakeholders can be seen beneath the job-seeker table:\ref{UsabilityEmployer}. 

% Please add the following required packages to your document preamble:
% \usepackage{longtable}
% Note: It may be necessary to compile the document several times to get a multi-page table to line up properly
\begin{longtable}{llll}
\caption{Key figures: Usability Tests w. Job-Seeker stakeholders (n=6)}
\label{UsabilityJob}\\
\hline
\rowcolor[gray]{0.9}
\multicolumn{4}{|c|}{\textbf{Key figures: Job-Seekers}} \\ \hline
\endhead
%
\rowcolor[gray]{0.9}
\multicolumn{1}{|l|}{\textbf{Average Time}} & \multicolumn{1}{l|}{Unit} & \multicolumn{1}{l|}{\textbf{Mistakes (\#)}} & \multicolumn{1}{l|}{Unit} \\ \hline
\rowcolor[gray]{0.9}
\multicolumn{1}{|l|}{\textit{All testers}} & \multicolumn{1}{l|}{\textit{(s)}} & \multicolumn{1}{l|}{\textit{All testers}} & \multicolumn{1}{l|}{\textit{Total(\#)}} \\ \hline
Task 1 & 66,33 & Task 1 & 2 \\ \hline
Task 2 & 100,50 & Task 2 & 2 \\ \hline
Task 3 & 84,17 & Task 3 & 5 \\ \hline
Task 4 & 139 & Task 4 & 8 \\ \hline
\rowcolor[gray]{0.9}
\multicolumn{1}{|l|}{\textit{Refu. \& Immi.}} & \multicolumn{1}{l|}{\textit{(s)}} & \multicolumn{1}{l|}{\textit{Refu. \& Immi.}} & \multicolumn{1}{l|}{\textit{Avg.(\#)}} \\ \hline
Task 1 & 94 & Task 1 & 0,5 \\ \hline
Task 2 & 177,50 & Task 2 & 0,5 \\ \hline
Task 3 & 128,50 & Task 3 & 0,5 \\ \hline
Task 4 & 279 & Task 4 & 1,5 \\ \hline
\rowcolor[gray]{0.9}
\multicolumn{1}{|l|}{\textit{Ethnic Danes}} & \multicolumn{1}{l|}{\textit{(s)}} & \multicolumn{1}{l|}{\textit{Ethnic Danes}} & \multicolumn{1}{l|}{\textit{Avg.(\#)}} \\ \hline
Task 1 & 52,50 & Task 1 & 0,25 \\ \hline
Task 2 & 62 & Task 2 & 0,25 \\ \hline
Task 3 & 62 & Task 3 & 1 \\ \hline
Task 4 & 69 & Task 4 & 1,25 \\ \hline
 &  &  &  \\ \hline
 \rowcolor[gray]{0.9}
\multicolumn{1}{|l|}{\textit{Danes / Refug.}} & \multicolumn{1}{l|}{\textit{N/A}} & \multicolumn{1}{l|}{\textit{Danes / Refug.}} & \multicolumn{1}{l|}{\textit{N/A}} \\ \hline
1 & 0,5585106383 & 1 & 0,5 \\ \hline
2 & 0,3492957746 & 2 & 0,5 \\ \hline
3 & 0,4824902724 & 3 & 2 \\ \hline
4 & 0,247311828 & 4 & 0,8333333333 \\ \hline
\end{longtable}

% Please add the following required packages to your document preamble:
% \usepackage{longtable}
% Note: It may be necessary to compile the document several times to get a multi-page table to line up properly
\begin{longtable}{llll}
\caption{Key figures: Usability Test w. Employers (n=2)}
\label{UsabilityEmployer}\\
\hline
\rowcolor[gray]{0.9}
\multicolumn{4}{|c|}{\textbf{Key Figures: Employer}} \\ \hline
\endhead
%
\rowcolor[gray]{0.9}
\multicolumn{1}{|l|}{\textbf{Average Time(s)}} & \multicolumn{1}{l|}{\textbf{Unit}} & \multicolumn{1}{l|}{\textbf{Mistakes(\#)}} & \multicolumn{1}{l|}{\textbf{Unit}} \\ \hline
\rowcolor[gray]{0.9}
\multicolumn{1}{|l|}{\textit{All Testers}} & \multicolumn{1}{l|}{\textit{(s)}} & \multicolumn{1}{l|}{\textit{Opgaver}} & \multicolumn{1}{l|}{\textit{Total (\#)}} \\ \hline
Task 1 & 239 & Task 1 & 0 \\ \hline
Task 2 & 61,5 & Task 2 & 1 \\ \hline
Task 3 & 191,5 & Task 3 & 1 \\ \hline
\end{longtable}

The overall goal of the method, was to answer the stated questions from the method sections, as well as generalizing the results if possible. The questions were: 

\begin{itemize}
    \item Is the system intuitive?
    \item Is there any difference in performance between refugees and ethnic Danes?
    \item Can people access the information they need in an optimal and correct manner?
\end{itemize}

The way we proposed to describe whether the system was intuitive or not, was to measure the number of questions asked needed to solve a task, and the number of errors in clicking. Low numbers in both measurements would indicate that a stakeholder finds a system intuitive. No threshold for what was a 'high' number of errors or questions was agreed on upfront, so a 'high' number will be defined as a relatively high number to the population of testers it comes from, and opposite for low numbers. 

Number of questions asked during the test-session was excluded from this table, due to inconsistencies in the data-collection process of this metric. So the only metric we have to conclude upon is the number of errors. No significant outliers can be detected for all testers, when you look at the results relative to the complexity. Ethnic Danes seem do be dis proportionally challenged in task 3 when you look at errors, due to reasons discussed in the contextual interview outcome. Beyond that no particular outliers can be detected from the errors.

Whether the system is intuitive in an absolute sense, we can not conclude upon due to a lack of baseline. But if we ask the question if the system is intuitive to refugees and immigrants, with the ethnic Danes performance as a baseline for evaluation, the conclusion seems to be that it is equally intuitive to both groups. 

There is a significant difference in performance when you look at the time measurements between ethnic Danes and refugees and immigrants, where the latter performs worse. In terms of errors, the absolute difference on the sum of averages for both groups is only 0.25 in favor of the ethnic Danes making less mistakes. Therefore it can be suggested that there is a difference in performance, in favor of the ethnic Danes. 

The last question is answered by a conjunction of both the time-measurement and the number of errors. The level of comprehension is proposed as high when time and number of errors is low. As we can see in the 'Danes / Refug.' columns, Ethnic Danes outperform refugees and immigrants on all measures except one. This indicates that ethnic Danes have an easier time accessing information on the platform, relatively to refugees and immigrants. 
\\

In terms of employers, a lot less work was put into analysing the results, since the scope of the test was limited to two of the questions, since there is no refugees and immigrants group within the employers. Since a baseline comparison is not possible for the employers, it is impossible to conclude upon any of the questions asked before the testing. 

In terms of sources for error, we want to highlight:

\newpage

\begin{itemize}
    \item Number of participants in tests
    \item Age versus Refugees and Ethnic Danes
    \item Style of task-solving
    \item Pre-existing knowledge regarding the prototype
\end{itemize}

The number of participants in both tests were low, due to problems with recruiting stakeholders and time before deadline. This hinders our possibility of generalizing our results, since the number of participants are too low to say that the results are representative. There was a significantly higher average age within the refugees and immigrants, which might average around mid forties, and the ethnic Danes averaged around mid twenties. If you were to divide the groups up in terms of age, with a group below thirty and one above, you could show the exact same difference in performance, between the younger group and the older group. The participants style of task solving was different, in the way the ethnic Danes, generally were less careful in regards to making mistakes, quickly interacting with the prototype. The refugees or immigrants were more thoughtful, and took longer time to process the pages they went through, without it necessarily being necessary for solving the task at hand. The relatively higher completion times for refugees might be a result of this approach. Some of the refugees had prior knowledge regarding the prototype, since they were also part of the co-creation workshop. This might also have inferred on the results of the test.

\subsubsection{Overall Analysis: Discover And Define 3. iteration}

The third round of Discover and Define, included a new employer, which brought more complexity to the employer persona. A formal update of the persona method was not done, due to time restrictions, but similar to the second iteration of Discover and Define, new dimensions of action is discovered. The employer came from a different industry than the other employer, and differed in numerous ways, as is highlighted in the figure below:[\ref{persona_update_2}]. 

\begin{figure}[H]
\caption{Employer Axis: Multi dimensional representation of employer actions}
\centering
\label{persona_update_2}
\includegraphics[width=\textwidth]{Images/Analysis & results/Persona_update_2.PNG}
\end{figure}

This realization brings a set of new design implications. The new employer, did not rely on the traditional information as CV's and motivated applications, which are both highly valued by the other employer. He preferred a phone conversation and legal documentation stating i.e that you have a permission to work in Denmark. This approach suits this employers rate of hiring, since it is fast paced, and some times the employer needs an employee with a days notice. 

We deviated from the prescribed way of doing the usability testing-method, by using the gathered data in the way we did. the method does not provide a framework for how to process the data gathered, only how to gather it. Therefore, we chose to use the key figures to conclude upon, if possible, and let the outliers be handled by the contextual interview, since the framework allows for asking more specific question to specific problems.   

The Usability Testing Method failed in presenting results which could be generalized, due to reasons stated in the section prior, but gave insights into problems which are to be dealt with in the third iteration of Develop and Deliver.

\subsection{Develop And Deliver}

\subsubsection{1. Iteration}

The following methods outcome will be covered from the first iteration:
\begin{itemize}
    \item Storyboarding
    \item Role-Playing
    \item Click-Through-Prototypes
\end{itemize}

The outcomes of the methods will be presented below.

\begin{itemize}
    \item \bf{Storyboarding:}
\end{itemize}

The two storyboards can be seen in figures: [\ref{story-wall}],[\ref{story-morph1}], and [\ref{story-morph2}]. In terms of the storyboard created digitally, more readable versions have been put in the appendix H: [\ref{H}].

\begin{figure}[H]
\caption{Storyboard 1: Based on Wall Walk}
\centering
\label{story-wall}
\includegraphics[width=\textwidth]{Images/Storyboards/Storyboard_wall_walk.png}
\end{figure}

\begin{figure}[H]
\caption{Storyboard 2.1: Based on Morphology}
\centering
\label{story-morph1}
\includegraphics[width=\textwidth]{Images/Storyboards/Storyboards Morfologi 1.jpg}
\end{figure}

\begin{figure}[H]
\caption{Storyboard 2.2: Based on Morphology}
\centering
\label{story-morph2}
\includegraphics[width=\textwidth]{Images/Storyboards/Storyboards Morfologi 2.jpg}
\end{figure}

The Storyboards allowed us to visualize our thoughts, and prepared us for creating the prototype based on the raw ideas for how it should look and work based on the storyboards.

\newpage

\begin{itemize}
    \item \bf{Role-Playing:}
\end{itemize}

The Role-playing did not infer on the interview structure itself, since no problems were seen during the run-through. As for the time, an estimate of a 1 hour interview was appropriate, since this was around the time the two sections took to go through when role-playing. 

\begin{itemize}
    \item \bf{Click-Through-Prototypes}
\end{itemize}

The outcome was two prototypes, one of the 'employer'-UI and one of the 'job-seeker'-UI. The prototypes are interactive in the sense that buttons can be pressed and present new pages, when using the Just-In-Mind application for running through the prototypes.

\begin{figure}[H]
\centering
\caption{Homescreen of 1. Iteration Click-Through Prototype, Applicant (left) and employer (right)}
\label{click-through1}
\adjustbox{trim={.0\width} {.15\height} {0.0\width} {.15\height},clip}
{\includegraphics[width=\textwidth]{Images/Click-Through/Proto1.jpg}}
\end{figure}

\subsubsection{Overall Analysis: Develop And Deliver 1. Iteration}

The general concerns when developing the 1. iteration prototype were:
\begin{itemize}
    \item Design for inclusivity
    \item User-oriented design vs. Disruption
    \item Simplicity vs. Complexity
    \item Designing UI on a concrete and general level
    \item Removing or minimizing Bias-enforcing information sources
\end{itemize}
These were tackled by focusing on the goal of the first iteration of the prototype, being an concrete interactable functional UI prototype that could be utilized to facilitate a deeper investigation of the problems faced by our target group in the job application process. To enable an easier testing cycle, with more constructive feedback, focus was on producing an inclusive simple prototype which could serve as a product, that shared many similarities with the traditional job application process, but still succeeded with the ambition of reducing the inherent bias in the process. The primary thought process behind this decision was the conviction that humans can easier grasp a concept and give constructive feedback and critique if it is familiar and relatable to them. 

The methods used throughout this diamond worked in conjunction with one another. This prototype development cycle was focused on creating a product that would be subjected to further development in the future, so preparation of a deliverable product was not in focus. The storyboarding process allowed us to structure the flow of the concept, to begin development of the click-through prototype in Just-In-Mind. This process also resulted in a realization of just how much time is wasted on both ends in the traditional job-application process, meaning that if The Seed can address this fundamental issue with the process and improve upon it, it would result in a significant value proposition for groups of employers whose focus is reducing the amount of man-hours spent searching for employees. 

The click-through prototype allowed for a rapid development of a functioning real-world prototype without the coding skill requirements and massive time investments required to build a platform. The resulting prototype addressed the factors leading to bias, previously identified throughout the literature review, by removing any chat functionality, codifying user data by making all profiles anonymous and reducing actions requiring writing capabilities in danish, by creating a short standardized version of a motivated application. Role-playing was used to validate the functionality and flow of the prototype, and to ensure it targeted our user group, before presenting it to the test users. These sessions gave insight into the essential need for user testing to address concerns which we were hypothesizing existed based on our discover and define phase.

It was made clear during this stage that several inherent UX issues existed within our initial concept, which would have to be addressed during the development of the 2. Prototype, and that 1 cycle would not have been sufficient to produce a satisfying concept. These concerns were kept in mind during the user testing of prototype 1 and were addressed if brought up by test users.


\subsubsection{2. Iteration}

The following methods outcome will be covered in the from the second iteration:

\begin{itemize}
    \item Co-Creation Workshop
\end{itemize}

The Following methods were updated due to input from the second iteration of Discover and Define:

\begin{itemize}
    \item Click-Through-Prototypes
\end{itemize}

The outcomes of the updated method and applied methods will be presented below.

\begin{itemize}
    \item \bf{Co-Creation Workshop}
\end{itemize}

The outcome of the session was insights, which will be translated into input to a new round of KJ technique, taking the insights and translating them into user-statements like: 'I the user would like *insert negative or positive input from the workshop*', in order to structure it through the KJ technique. Another outcome of the Co-creations were tangible data which was ready to implement as features, not needing to go through the KJ-technique. In terms of actual development outcomes, The stakeholders produced a number of tables, ranking their priorities within the areas of sorting and material for presenting a company. The stakeholders also produced in pairs, a visual suggestion for how a notification-page should look and function.

\begin{itemize}
    \item \bf{Click-Through-Prototypes}
\end{itemize}

New features were added, and based on the feedback given features were altered so they were in accordance with the users views.
\begin{figure}[H]
\centering
\caption{Home screen of 2. Iteration Click-Through Prototype, Applicant (left) and employer (right)}
\label{click-through1}
\adjustbox{trim={.0\width} {.15\height} {0.0\width} {.15\height},clip}
{\includegraphics[width=\textwidth]{Images/Click-Through/Proto2.jpg}}
\end{figure}

\subsubsection{Overall Analysis: Develop And Deliver 2. Iteration}

The general concerns when developing the 2. iteration prototype were:
\begin{itemize}
    \item CV trade-off: Employer value vs. The Seed disruption
    \item Thorough employer vs. Impatient employer
    \item Highly integrateable vs. less integrateable
\end{itemize}

Several key issues were identified throughout the co-creation workshop, which were than addressed, in conjunction with the previously stated concerns, in the development of the 2. Iteration of the click-through-prototype. The issues were: fundamental UX issues that were partly identified throughout the previous development cycle, wording issues due to the language barrier, wishes amongst the applicants to become more personally acquainted with the workflow of the job and a demand from the employer to have access to more traditional CV information such as work experience and educational background. The 2. Iteration is an amalgamation of functionality to address these issues and concerns, a continual focus on achieving The Seeds mission statement of reducing the bias in the process and a wish to create a prototype with the ability to perform real-world scenario tests, such as being able to select an applicant to call-in for a job-interview, respond to a call-in, reject an applicant after the job interview, and find relevant information about a job interview. 

The realization that the structure of the CV itself creates bias, due to the lack of work experience often associated with our target group, meant that we had to change our approach to the presentation of applicants. How can we reduce the bias and codify data in the CV, and push it aside in favor of the matchmaking done by The Seed, when employers value a CV so highly? Dates, years, and employment duration was removed from the CV, and the CV itself was deprioritized in the presentation of the applicants to the potential employer, in order to nudge them into selecting applicants based on the matchmaking, it was however not hidden, as this could lead to distrust in the system.

With the data collected from the contextual interviews and workshop, a more streamlined version of the prototype could be designed with a larger focus on future implementation into a danish context, where the product must be designed for different types of employers and job seekers. A concrete adaptation was the request from users to receive notifications about upcoming job interviews as a text message in the same way the danish dentists sends out texts a day before an appointment. Another adaptation was the creation of a searchable activity history allowing the danish Job centers to gain access to a job-seeker’s application history to ensure they fulfill the requirements to receive unemployment benefits.


\subsubsection{3. Iteration}

The Following methods were updated due to input from the third iteration of Discover and Define:
\begin{itemize}
    \item Click-Through-Prototypes
\end{itemize}

The outcomes of the update will be described below.

\begin{itemize}
    \item \bf{Click-Through-Prototypes}
\end{itemize}

The outcome of the update is a finished prototype, where the features were implemented, and the feature which were not implemented were presented as future work possibilities to The Seed. A run through of the flows possible in the 3rd prototype can be found in appendix F [\ref{F}] (applicant) and appendix G [\ref{G}] (employer).
\begin{figure}[H]
\centering
\caption{Example of flow in 3rd iteration prototype (applicant side)}
\label{click-through1}
\adjustbox{trim={.0\width} {.0\height} {0.0\width} {.0\height},clip}
{\includegraphics[width=\textwidth]{Images/Click-Through/applicantflow3.jpg}}
\end{figure}

\subsubsection{Overall Analysis: Develop And Deliver 3. Iteration}


The problematic aspects of the UX identified through the usability tests were addressed to develop a 3. Iteration of the click-through-prototype, which would function as a final concept for the eventual UI implementation into the overall platform. UX optimization was the focus, to reduce the identified performance differences between ethnic Danes and refugees/immigrants, by optimizing towards this group. A concrete change made to the prototype in response to feedback from immigrant testers who made more mistakes in the tasks was the separation of the “Job match” functionality from the “Active applications” functionality, making these two separate buttons on the Home screen. Timestamps, dates, and company names were attached to all activities to allow for an easier overview.\\

Through the development of the final prototype we realized the significant importance of QoL regarding the platform. We can´t change the implicit biases of people with one job application platform, nor can we change the systemic biases that exist in danish society, so how can we hope to reduce it? If we can create a platform which is so time saving, convenient and effective to use that any employer would greatly benefit from it, then we can slowly reduce the bias in the system over time. The QoL improvements, which were addressed in this final prototype is the value proposition to the employers. To create a significant systemic change, the platform must be in use by a significant number of employers and job seekers.

discussion: 
Discussion:
\subsection{The perception of the research question}

We set out when the project started, to design a mobile interaction tool aimed at reducing bias in the job-seeking process. In order to fully grasp, whether that has been achieved, and can be generalized, a relevant questions emerge, What is bias? \\

As the project has progressed, we have been in running contact with The Seed, regarding the progression of our project and the technology we are developing for them. In these talks, the question of bias, and what causes bias has been a topic of discussion. As we presented the technology, comments were made regarding the information we chose to present, and discussions, whether or not it could potentially cause bias, occurred. We had similar conversations with the stakeholders whom we interviewed, whom all had their own opinions on what information is 'fair', and what information causes bias. The literature presented a number of biases, both systemic and implicit, and the information sources which feeds bias. When you ask what causes bias, you get different answers depending on who you ask. Danish discriminatory legislation has one view on which basis you can charge, convict and sentence an individual for discrimination, but the institute of human rights, whom are more progressive, have a broader view on what discrimination is. There is no apparent ground truth on what bias is. So the question is, where do we draw the line?\\

So when questioning whether or not we can generalize our results, the definition of bias will have to be similar if someone were to reproduce this study. Since our definition of bias, is not only dependent on literature, but also the context and stakeholders of the project, an absolute generalization is impossible. \\

The underlying intention of proposing the research question, is not to just eliminate all information from the process, which would be blind recruitment, effectively eliminating any chance of bias inferring on decision making. The intention is to balance the trade-off between stakeholder value, and systemic changes, in the pursuit of having a more ethical and fair process. In balancing the trade-off, one must decide, which stakeholders to provide value for. User-oriented design is the bedrock of the Double Diamond framework, and therefore dictates the job-seeker and the employer as the relevant stakeholders to design for. During our conversations with The Seed, our design was influenced by their view on what bias is, and made us change features based on their view. This means that our results are more context-dependant, than if we had chosen not to let it influence the design. Our ability to generalize is therefore worsened. \\

The contexts we have introduced hinders a successful generalization of results, but also means that the solutions produced are well fit for the context which influenced the design. In cooperation with The Seed, the design has been formed by their vision, market-factors, and the level that their technology is at, at the time of writing. So although, these context have interfered with the scientific purity of the project, it increases the possibility of being useful to The Seed, and eventually being a part of the change they would like to make in the world. By allowing these factors to matter, we believe that we have made a reformative suggestion for how to solve the problem of bias. A solution which seeks to compromise on ideals, and produce value in the eyes of the stakeholders, while being a step forward towards a just process, where people are judged on merits. 

\subsection{The influence of Covid-19}
Due to the nature of the projects method-selection being dependent on face-to-face stakeholder interaction throughout the project-phases, Covid-19 posed a risk of influencing the level to which we could perform these methods. This could be either by government regulation, if stakeholders had health concerns, or if they had loved ones whom they would not put at risk by exposing themselves to others. The project was aware of this risk from the beginning of the project, and therefore mitigated the risk by making procedural frameworks for digital versions, of the methods which where intended to be face-to-face. These frameworks were tested on a couple of subjects, and the following general issues were recorded, which all could have a potential negative impact on the quality of the output:

\begin{itemize}
    \item Unstable internet connection
    \item Unstable hardware / poor quality of hardware
    \item Speech-delay (lag)
    \item Problems with platform registration for co-creation
    \item Problems with usage of platform functions
\end{itemize}

We chose the call-function in Zoom as our communication technology of choice, due to us being familiar with it, and the ability to create breakout rooms, with minor groups which could discuss a named topic for a limited time. For brainstorms and creative processes, we chose Miro, due to the ability of having numerous stakeholders working with the same dashboard at the same time. \\

The unstable internet connection could mean a number of things, either causing the stakeholder to be unable to connect to the zoom-meeting or causing excessive speech-delay. The unstable hardware could either be an old computer, not having the necessary processing power to run the functions needed to facilitate the method, or bad microphones, making communication unclear or sometimes impossible. The speech-delay in itself, can cause communication to be cumbersome and with a lot of friction, due to the time it takes from you ask a question to the stakeholder hears it, can be several seconds. We experienced on numerous occasions that we asked the stakeholder if they heard us, but they had not received the question through their headset yet. This has a bad influence on the chemistry, and sense of flow in the conversation, either prolonging the time the interview takes, or as one stakeholder communicated, making someone less inclined to answer extensively. In order to use both use Miro and Zoom for the first time, you would have to either download and register, or just register in order to access the technologies. Some stakeholders were reluctant with registering, due to it being excessive or cumbersome, for people who are not skilled within IT. Stakeholders who are unfamiliar with using i.e screen-sharing on zoom, could find it difficult during a session, and ruin the flow of a method, or waste other stakeholders time. Some of these problems could be mitigated and the following section presents a selection of mitigation-strategies for some of the bullets.

\subsubsection{Risk mitigation}

In order to mitigate the risk of the bullets further up having an impact on the output of the digital methods, we came up with the following mitigation strategies:

\begin{itemize}
    \item Preparation e-mail
    \item Pre-interview as a test and preparation for co-creation
\end{itemize}

The preparation e-mail was sent out a couple of days before the digital method was to be executed with the stakeholder. In the preparation mail, a small presentation of the agenda was given, as well as a guide for using the communication technologies for first time users. This included a link for registering, as well as bullets on how to use the functions we were going to use during the digital method. The stakeholder was asked to register before the methods execution, in order to minimize time-loss. This strategy minimizes the risk of stakeholders having trouble with registering and using functions within the technologies they are due to use in the execution of the method.
During the second iteration of designing, stakeholders would go through both a contextual interview, and a part take in a co-creation workshop along with other stakeholders. The contextual interview was used strategically to introduce the communication technologies, even though they were not a part of the execution of the contextual interview. This would minimize the risk of problems either occurring with registering or using the technologies when all stakeholders are gathered, and more time can be wasted.
This leads to the actual implications which the digital methods had, which will be discussed in the next section.

\subsubsection{Execution: The actual influence of executing digitally}

The digital methods were only executed during the second and third iteration of design. in the third iteration, in the form of a digitally executed contextual interview and Usability test with the paper-prototypes method as a border-object. And in the second iteration a digitally conducted contextual interview, with a paper-prototype as border-object. Since time-taking was a performance measure for the stakeholders, and explanation time of each assignment was a part of the total time, the extra time in the digital format influenced the outcome of all the tasks total time. The Contextual interview in the second iteration, had both hardware issues in a microphone of poor quality, and a lagging connection, making communication hard. This prevented us from capturing all the data we wanted in the pre-interview, since the stakeholder had a time-cap, and we needed to prioritise our time, to get the most relevant data. The experience from our point of view, was that the interview chemistry was compromised, due to the interruptions, falling in and out of connection, and the numerous times questions were repeated. 

\subsubsection{Practical implications of Covid-19 on project timeline}

We were originally scheduled to do a co-creation workshop with stakeholders from the job-seeker segment, the 25Th of October. The 23rd of October, new governmental restrictions was introduced, introducing the risk of cancellation or postponing of the workshop, and the 24Th of October, the workshop was postponed. The timeline remained unchanged, but left us with a workweek to complete a full Define-, Develop-, and Deliver-phase, which, as it was stated in the above sections, forced us to streamline the process, deselecting methods from the phases. 

\subsubsection{Overall assessment of Covid-19 impact}

With the above sections in mind, we will assess that the impact on the project has been minimal. There was no impact on project timeline, and project quality was affected in a minimal manner. The output produced in the processes mentioned in the executed digital methods, was influenced, but due to other factors the output used, did not include the output produced by the part influenced in a negative way by doing it digitally. 

discussion: 

discussion: 
